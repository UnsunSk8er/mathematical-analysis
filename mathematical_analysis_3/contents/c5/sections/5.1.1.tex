\subsection{含参变量积分的概念}

\begin{definition}[含参变量的积分]
    $f(x,y)$在$[a,b]\times[c,d]$有定义,且对每个固定的$y\in [c,d]$,关于$x$的函数$f(x,y)$在$[a,b]$上R可积.令
    \[I(y)\coloneq\int_a^b f(x,y)\operatorname{d}x,\quad y\in [c,d]\]
    称为含参变量的积分,其中$y$是参数.它对应数列或函数列中的变数$n$.
\end{definition}

\begin{remark}
    $I(y)$显然是$y$的函数.
\end{remark}

\begin{definition}[一致收敛极限]
    设$y_0\in[c,d]$,若存在函数$\varphi(x),x\in [a,b],$使得$\forall\varepsilon >0,\exists\delta >0,\forall y\in \check{N}(\delta,y_0):$
    \[|f(x,y)-\varphi(x)|<\varepsilon,\quad\forall x\in[a,b]\]
    则称当$y\to y_0$时,$f(x,y)$在$x\in[a,b]$上一致收敛于$\varphi(x)$.
\end{definition}

\begin{remark}
    若$f(x,y)$为$[a,b]\times[c,d]$上的连续函数,则由紧集上连续函数的一致连续性可知,当$y\to y_0$时,$f(x,y)$在$x\in [a,b]$上一致收敛于$f(x,y_0)$.
\end{remark}

以下讨论一致收敛的极限函数的性质(参照一致收敛函数列的性质).

\begin{theorem}[极限函数一致收敛的Cauchy准则]
    $\lim_{y\to y_0}f(x,y)$在$[a,b]$上一致收敛当且仅当$\forall\varepsilon,\exists\delta >0$,当$y_1,y_2\in\check{N}(\delta,y_0)$时,
    \[|f(x,y_1)-f(x,y_2)|<\varepsilon,\quad\forall x\in[a,b].\]
\end{theorem}

\begin{proposition}[极限函数连续的充分条件]
    对每个固定的$y\in[a,b],f(x,y)$都是关于$x\in[a,b]$的连续函数,若极限函数
    \[\lim_{y\to y_0}f(x,y)=\varphi(x)\]
    在$[a,b]$上一致收敛,则$\varphi(x)$是$[a,b]$上的连续函数.
\end{proposition}