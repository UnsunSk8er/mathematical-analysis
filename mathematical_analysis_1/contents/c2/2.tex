\chapter{实数理论与数列极限}

这章主要讨论如何从有理数集构造实数集,构造完成后将用数列作为工具研究实数集.

\begin{note}实数集与自然数集、整数集、有理数集的区别在哪里?构造实数集的动机是什么?

在集合论中,由集合和映射出发,构造出了自然数集,它是可数的.
由等价关系和自然数集上的加法乘法运算出发,从自然数集构造了整数集,再从整数集构造了有理数集,它们也是可数的.
有理数集具有稠密性,它的元素虽然已经很多,但仍然有间隙,是“离散”的数集.

对于实数集的构造,本质上区别于前三个数集的构造,即要从“离散”的有理数集构造出一个“连续”的数集.

不可数个数组成的集合和至多可数个数组成的集合的区别,就像在纸上画一条线不需要提笔,和在纸上按这条线的轨迹画至多可数个点至少需要提笔一次的区别一样.从这样的类比来看,实数集完全区别于有理数集、整数集和自然数集.

但从乘方逆运算的角度,却和整数集、有理数集的构造有动机上的联系.要从“不完备”的数集构造出一个“完备”的数集.

整数集的构造,从加法逆运算的角度,将加法逆元素纳入自然数集.有理数集的构造,从乘法逆运算的角度,将乘法逆元纳入整数集.而实数集的构造,从乘方逆运算的角度,是要将乘方的逆元纳入有理数集中.
\end{note}

\begin{remark}
    虽然从“离散”和“连续”的角度,实数集和前三个数集是本质不同的,但仍然有些联系,即它们都是无限集.这给我们提供了研究实数集(不可数集)的切口——数列(可数集).
\end{remark}

\begin{remark}
    这里数的乘方保证了乘方逆运算作用在一个非负数上,而对于负数,涉及到复数集.数学分析中不予讨论.%//INFO(笔者也还不知道复数集和它们的联系,哈哈)
\end{remark}

\begin{note}实数的构造方法有哪些?用数列极限来研究实数集的原因是什么?怎么研究?

    实数构造的方法有Dedekind分割、无限十进制小数、闭区间套、Cauchy列、无穷级数等,本讲义选用Dedekind分割.这些构造方法的核心思想是“可数无限“的思想,数列极限的核心思想也是“可数无限”.

    我们将从Dedekind定理、确界定理、Heine-Borel定理三个角度阐述实数的完备性,然后从数列角度介绍单调有界定理、闭区间套定理、Bolzano-Weierstrass定理、Cauchy收敛原理刻画实数的完备性,最后证明七个完备性定理是等价的.
    
\end{note}

\begin{remark}
    实数理论形成于19世纪末,完善于20世纪初,理解难度远远大于16、17世纪牛顿莱布尼兹时代建立的微积分.
\end{remark}

\section{Euclid空间中点列的收敛性}

\subsection{无理数的历史}

\begin{example}
    证明二次方程$x^2=2$没有有理根.
    \begin{proof}
        即证明$\sqrt{2}$不是有理数.
        用反证法.假设$\sqrt{2}\in\Q$,则$\sqrt{2}=\frac{q}{p}$,其中$p,q\in\Z$互素且$p\ne 0$.从而
        \[\z(\frac{q}{p}\y)^2=2\iff q^2=2p^2.\]
        于是$q$为偶数,从而$q=2k,k\in\Z$,于是
        \[(2k)^2=2p^2\iff p^2=2k^2.\]
        这说明按假设$p$也是偶数,这与$p,q$互素矛盾.假设不成立,从而命题得证.
    \end{proof}
\end{example}

\begin{remark}
    这个例子说明,$\Q$中仍然有间隙,是不连续的.
\end{remark}

\begin{note}关于超越数与实数的构造
    
    虽然$\sqrt{2}$不是有理数,但它还是一个代数方程的根.而更多无理数是不能表示成代数方程的根的,比如圆周率$\pi$,因为代数方程终归是“有限的游戏”.虽然如此,代数方程可以表示许多无理数,“有限的游戏”构造出“无限的结果”,这是多么美妙.
    
    而像圆周率$\pi$这样能表示成代数方程的根的无理数,称为超越数,结合上面的讨论知道,它不能通过有限次运算得到.因此更多的无理数,像超越数不能用有限的过程来构造,只能通过无限的过程!这就表明,实数集的构造用整数集和有理数集的构造方法是行不通的.即若用整数集和有理数集的构造方法构造实数,是构造不出超越数的.
\end{note}

\begin{remark}
    上面的讨论说明,有限次的乘方逆运算,并不能囊括所有的无理数!
\end{remark}%$\R^n$的线性结构
\subsection{Dedekind分割}

\begin{definition}
    [Dedekind分割]
设数集$K$的一个划分$\{\alpha,\beta\}$,若满足:
    \begin{enumerate}
        \item $\alpha$“向下封闭”:$\forall x,y\in K,x<y:y\in \alpha\implies x\in \alpha.$
        \item $\alpha$“无最大元”:$\forall x\in\alpha,\exists y\in\alpha\st y>x.$
    \end{enumerate}
    则称该划分为$K$上的一个Dedekind分割(Dedekind cut),记作$\alpha\mid\beta$.其中$\alpha$称为这个Dedekind分割的下集(lower set),$\beta$称为这个Dedekind分割的上集(upper set).
\end{definition}

\begin{remark}
    $\{\alpha,\beta\}$是$K$的一个划分,于是$\alpha,\beta$满足:
    \begin{enumerate}
        \item $\alpha,\beta\ne\emptyset$
        \item $\alpha\cap\beta =\emptyset$
        \item $\alpha\cup\beta =K$
    \end{enumerate}
\end{remark}

\begin{remark}
    也可以规定$\beta$“向上封闭”且“无最小元”,本质是一样的,只是后面的证明会因为定义而改变,只要讲法自洽都是可以的.
\end{remark}

\begin{definition}
    [实数集]
    有理数域$\Q$上的所有Dedekind分割的下集所组成的集合称为实数集(set of real numbers),记作$\R$.其中每个Dedekind分割的下集表示一个实数(real number).
\end{definition}

\begin{note}
    这个下集确定的实数也可以看作是这个Dedekind分割或者上集.从而实数就定义为一个Dedekind分割或者它对应的上集或者它对应的下集.
\end{note}

    Dedekind分割体现了一个“无限过程”,实数本质上是一个“无限过程”.
    
    由于$\alpha$中无最大元,且$\Q$具有稠密性,于是任取一个数$a_0\in\alpha$总能取到一个数大于它,记这个数为$a_1$,反复进行下去,我们就得到了一列有序的数字$a_1,a_2,\cdots$(后面我们将会定义数列).直观上这列数是越来越大的,但又不能大到跑出$\alpha$,于是这列数会逼近一个数,即我们要定义的实数.

\begin{note}
    这列数所逼近的数可能在$\Q$中,此时即为$\beta$的最小元,也可能不在$\Q$中,此时这个数是个无理数.
\end{note}

\begin{note}
    这列数越来越大与后面的数列单调性有关,不能大到跑出$\alpha$,与后面的有界性有关,这两者结合可以联想到数列的单调有界收敛定理,这是后话.
\end{note}
%$\R^n$空间的内积,范数和度量
\subsection{$\R^n$中点列的收敛性}

\begin{definition}[$\R^n$中点列的极限]
    
\end{definition}

\begin{definition}[邻域]

\end{definition}

\begin{definition}[有界点列]
    
\end{definition}

\begin{definition}[子列和极限点]
    
\end{definition}

\begin{proposition}[点列极限的简单性质]
    
\end{proposition}%$\R^n$中点列的收敛性%//INFO 实数域的构建及其结构

\section{数列极限的定义和性质}
\subsection{邻域和极限的简单性质}
\begin{definition}[邻域]
    给定$a\in\R$,令
    \[N_r(a)\coloneq\{x\in\R:|x-a|<r\},\quad \forall r\in \R^*.\]
    称$N_r(a)$为以$a$为中心,$r$为半径的邻域(neighbourhood),在不强调半径的情况下可记作$N(a)$.
\end{definition}

\begin{remark}
    在$\R$中,容易知道$N_r(a)=(a-r,a+r)$,然而我们没有用开区间直接定义,而是用度量对邻域进行定义.
\end{remark}

邻域在数学分析中用于描述“局部概念”十分便利.下面用邻域的观点叙述数列极限的定义.

\begin{theorem}
    设数列$\{a_n\}$,\\
    (1)$\{a_n\}$收敛于$a\in\R\iff n$充分大时$\{a_n\}$的各项都落在任一给定的邻域$N_{\varepsilon}(a)$中.\\
    (2)$\{a_n\}$收敛于$a\in\R\iff a$的任一邻域外都只有$\{a_n\}$的有限项.
\end{theorem}

\begin{proof}
    必要性按数列极限的定义显然成立.
    任取$\varepsilon>0$,则$N_{\varepsilon}(a)$外只有$\{a_n\}$的有限项,设其中下标最大的那个为$a_N$,则当$n>N$时,$\{a_n\}$全都落在了$N_{\varepsilon}(a)$中.此即按数列极限定义得证.
\end{proof}

\begin{remark}
    (1)注意,若$a$的任一邻域$N(a)$内都有$\{a_n\}$的无穷多项,不能推出$\{a_n\}$收敛.只能说有极限点.
    (2)可视为数列极限的“拓扑定义”.
\end{remark}

邻域是一个拓扑概念,从邻域的角度思考可以使问题更加直观.极限是一个无穷的概念,上述定理的第(2)条,使得我们将无穷问题转化为有穷问题来解决,这使得问题变得简单易操作了.%//INFO 邻域和极限的简单性质
\subsection{极限的运算法则}
\begin{theorem}[夹逼定理]\label{sandwich theorem}
    设数列$\z\{a_n\y\},\z\{b_n\y\},\z\{c_n\y\}$满足
    \[a_n\leqslant b_n\leqslant c_n,\quad n=1,2,\cdots.\]
    若$\lim_{n\to\infty}b_n=\lim_{n\to\infty}c_n=a$,则$\lim_{n\to\infty}b_n=a.$
\end{theorem}

\begin{remark}
    夹逼定理又被称为三明治定理(sandwich theorem)
\end{remark}

\begin{proof}
    \\Step1.按数列极限的定义对$\{a_n\},\{c_n\}$,易知$\forall \varepsilon >0,\exists N_1,N_2\in\N^{*}\st
    (\forall n(n>N_1):a_n\in N_\varepsilon(a))\land(\forall n(n>N_2):c_n\in N_\varepsilon(a))$.\\
    Step2.又由数列$\{a_n\},\{b_n\},\{c_n\}$满足$a_n\leqslant b_n \leqslant c_n,n=1,2,\cdots$.\\
    Step3.于是当$n>\max\{N_1,N_2\}$时,成立$a-\varepsilon <a_n\leqslant b_n\leqslant c_n < a +\varepsilon$.这说明$\forall\varepsilon >0,\exists N=\max\{N_1,N_2\},\forall n(n>N):|b_n-a|<\varepsilon$,此即按数列极限的定义得证.

\end{proof}%//INFO 极限的运算法则%//INFO 数列极限的定义和性质

\section{数列的收敛判别法}
\subsection{数列的上极限和下极限}


数列的所有极限点组成的集合非空,即可以是有限集、可数集、不可数集.
从数列来看,所有子列的极限刻画了数列的趋势,这个趋势多数情况是个范围.
我们关心的数列极限,直观上当这个范围越来越小时,就有可能存在.
既然我们只关心数列的极限,那么这个范围里面的具体元素并非我们所关心的.
于是为了描述控制数列趋势的这个范围,我们引入数列的上下极限的概念.
从而,当范围越来越小时,数列极限可能被上下极限“夹逼”出来.

\begin{note}
    思想方法类比数列极限的夹逼定理\ref{sandwich theorem},类似的思想还将在连续函数的刻画和Riemann可积的等价刻画中再度出现.
\end{note}

\begin{definition}[数列的上极限和下极限:定义方式1]
    设数列$\{a_n\}$,记$\{a_n\}$的所有极限点组成的集合为$E$,其中可以包含$+\infty$或$-\infty$.令
    \[\limsup_{n\to\infty}a_n\coloneq\sup E,\qquad\liminf_{n\to\infty}a_n\coloneq\inf E.\]
    称$\limsup_{n\to\infty}a_n$为$\{a_n\}$的上极限(limit superior),$\liminf_{n\to\infty}a_n$为$\{a_n\}$的下极限(limit inferior).
\end{definition}

\begin{note}
    从定义不难看出它是由确界原理和列紧性定理确保的.
\end{note}

$E\ne \emptyset$,即说明任意数列都存在极限点,从而存在上下极限,这使得极限点比极限对数列趋势的刻画更容易操作和把握.

关于如何求数列上下极限没有固定的简单方法.对只有有限个极限点的情况,可以求出所有极限点,从而看出上下极限,它们分别是最大的和最小的极限点,下举两例.

%//TODO 例
\begin{example}
    设数列
    \[\{a_n\}=\frac{(-1)^n}{1+1/n},\quad n=1,2,\cdots.\]
    求$\limsup_{n\to\infty}a_n,\liminf_{n\to\infty}a_n.$
\end{example}

%//TODO 例
\begin{example}
    设数列
    \[a_n=\frac{n^2}{1+n^2}\cos\frac{2n\pi}{3},\quad n=1,2,\cdots.\]
    求$\limsup_{n\to\infty}a_n,\liminf_{n\to\infty}a_n.$
\end{example}

\begin{note}
    按极限点的不同可以对极限点集$E$进行划分.
\end{note}

若数列有无穷多个极限点,则上极限未必是最大极限点.
这是由于上极限是极限点集$E$的上确界,上极限未必属于$E$.
下面的命题讨论了这个问题.

\begin{proposition}
    设数列$\{a_n\}$,$E$依然是这个数列所有极限点的集合.则
    \[\limsup_{n\to\infty}a_n\in E,\qquad\liminf_{n\to\infty}a_n\in E.\]
\end{proposition}

%//TODO 证明
\begin{proof}
   即证$\sup E\in E,\inf E\in E$.确切地说就是要证明$\{a_n\}$的极限点集$E$的上下确界是极限点,从而只需要证明存在子列收敛到极限点集$E$的上下确界.这里只证上极限的情况,下极限情况类似.
\begin{enumerate}
    \item 若$\limsup_{n\to\infty}a_n=+\infty$,则$E$无上界,即$\{a_n\}$的极限点集没有上界.
    这说明存在极限为$+\infty$的$\{a_n\}$的子列,于是$+\infty$是一个极限点,从而它在$E$中.
    \item 若$\limsup_{n\to\infty}a_n=-\infty$,由于$E$非空,则$E$中只有$-\infty$,从而它在$E$中.
    \item 若$\limsup_{n\to\infty}a_n=a\in\R$,这只需证明$\{a_n\}$有一个子列收敛到$a$,这样$a$作为这个子列的极限点必然属于极限点集$E$.
    由于$a$是$E$的上确界,于是在$a$的邻域$N_{\varepsilon_1}(a)$可以找到一个极限点$l_1$,存在$l_1$的邻域$N(l_1)\subset N_{\varepsilon_1}(a)$含$\{a_n\}$的无穷多项,从中选取一项$a_{k_1}$.
    同理可以找到极限点$l_2\in N_{\varepsilon_2}$,存在$l_1$的邻域$N_(l_2)\subset N_{\varepsilon_2}(a)$含$a_n$的无穷多项,从中选取一项$a_{k_2}$.
    这样由数学归纳法,做下去得到一个数列$a_{k_n}$,令$\varepsilon=1/n$,从而$a_{k_n}\in N_{1/n}(a)$,即
    \[a-\frac{1}{n}<a_{k_n}<a+\frac{1}{n},\quad n=1,2,\cdots.\]
    由数列极限夹逼性定理,得到
    \[\lim_{n\to\infty}a_{k_n}=a.\]
\end{enumerate}
综上所述,命题得证.
\begin{note}
    关于$l_n$为什么可以取到:$a$是$E$的上确界,意味着$\forall x\in E,x<a$,且$\forall \varepsilon>0,\exists x\in E,x>a-\varepsilon.$证明中的$\varepsilon_n$取为$1/n$,这是为了凸显$\varepsilon$任意小,使它趋于$0$以构造子列完成夹逼,夹逼出来的子列即为我们想要的.
\end{note}
\end{proof}

\begin{remark}
    上面的命题告诉我们,上下极限都是数列的极限点,进一步讲,上极限就是数列的最大极限点,下极限就是数列的最小极限点
\end{remark}

由上下极限定义由确界原理确保的部分,我们容易得出下列命题.

\begin{proposition}\label{sjxxydyxjx}
    设数列$\{a_n\}$,有
    \[\limsup_{n\to\infty}a_n\leqslant\liminf_{n\to\infty}a_n,\]
    等号成立当且仅当$\{a_n\}$极限存在.
\end{proposition}

\begin{proof}
    由上下确界的定义,不等式显然成立.
    现在看等号成立的情况.
    当$a_n$极限存在时,设极限为$a$.则极限点集为$E=\{a\}$.由于上下极限都属于$E$,从而它们都为$a$,于是上下极限相等.
    反之,上下极限相等,说明$E$中有且仅有一个极限点,这就说明$\{a_n\}$极限存在.
    此即得证.
\end{proof}

\begin{note}
    证明极限存在只需证明$\limsup_{n\to\infty}\leqslant\liminf_{n\to\infty}a_n$
\end{note}

数列极限可以用$\varepsilon -N$(或邻域)刻画,也可以看作上下极限相等的特殊情况.

当数列$\{a_n\}$极限为$a$时,极限唯一且$\forall \varepsilon,\delta>0,(a-\varepsilon,a+\delta)$中有数列的无穷多项,此区间外只有有限项.这是特殊情况.一般的情况是,当数列的上下极限存在且分别为$l,L$时,$(l-\varepsilon,L+\delta)$中有数列的无穷多项,此区间外只有有限项.因而上下极限也可由$\varepsilon -N$和邻域来刻画.

\begin{theorem}[数列的上极限和下极限:定义方式2]\label{sxjxdy2}
    设数列$\{a_n\}$,令
    \begin{align*}E_1=\{\exists a\in\R:\forall\varepsilon>0,\exists N\in\N^*\st\forall n>N:a_n<a+\varepsilon\},\\
    E_2=\{\exists a\in\R:\forall\varepsilon>0,\exists N\in N^*\st\forall n>N:a_n>a-\varepsilon\}.\end{align*}
    则
    \[\limsup_{n\to\infty}a_n=\inf E_1,\qquad\liminf_{n\to\infty}a_n=\sup E_2.\]
\end{theorem}

\begin{proof}
    只证明上极限的情况,下极限的情况类似.设$\limsup_{n\to\infty}a_n=L.$
    \begin{enumerate}
        \item 证明$L\geqslant\inf E_1.$用反证法,假设$L<\inf E_1$,即$L\notin E_1$,则$\exists\varepsilon>0,\forall N\in\N^*\st\exists n>N:a_n\geqslant a+\varepsilon.$这表明在$(L+\varepsilon,+\infty)$中有数列的无穷多项,于是由Bolzano-Weierstrass定理,在$(L+\varepsilon,+\infty)$中必然存在一个大于$L$的极限点,这与$L$是极限点集的上确界矛盾.因此假设不成立,于是$L\geqslant \inf E_1$
        \item 证明$L\leqslant\inf E_1.$用反证法,假设$L>\inf E_1$.即$\exists a\in E_1\st a<L.$则$\exists\varepsilon >0\st a+\varepsilon <L.$由于$a\in E_1$,于是$\exists N\in\N^*\st\forall n>N:a_n<a+\varepsilon.$这表明$(a+\varepsilon,+\infty)$中有$\{a_n\}$的有限项,因此$L$不是极限点,与$L$是上极限矛盾.因此假设不成立,于是$L\leqslant \inf E_1$
    \end{enumerate}
    综上所述,命题得证.
\begin{note}
    \begin{enumerate}
    \item Bolzano-Weierstrass定理表明,$\R$中任一数列都有极限点.
    \item 这里的上下确界即可看作$E$中最大最小值,这是由\ref{sjxxydyxjx}确保的.
    \end{enumerate}
\end{note}
\end{proof}

\begin{remark}
    讲上下极限为无穷的情况也纳入到定义内,令
    \begin{align*}
        E_1=\{\exists a\in \overline{\R}:\forall \xi >a,\exists n\in\N^*\st\forall n>N:a_n<\xi\}\\
        E_2=\{\exists a\in \overline{\R}:\forall \xi <a,\exists n\in\N^*\st\forall n>N:a_n>\xi\}
    \end{align*}
    则
    \[\limsup_{n\to\infty}a_n=\inf E_1,\qquad\liminf_{n\to\infty}a_n=\sup E_2.\]
\end{remark}

利用上下极限的$\varepsilon-N$(或邻域)刻画可以证明上下极限的保序性.

\begin{proposition}[上极限和下极限的保序性]
    设$\{a_n\},\{b_n\}$,若$\exists N\in\N^*\st \forall n>N:a_n\leqslant b_n$,则
        \begin{align*}
            &(1)\quad\limsup_{n\to\infty}a_n\leqslant\limsup_{n\to\infty}b_n,\\
            &(2)\quad\liminf_{n\to\infty}a_n\leqslant\liminf_{n\to\infty}b_n.
        \end{align*}
\end{proposition}

\begin{proof}
    只证明(1),(2)类似可证.设$\limsup_{n\to\infty}a_n=A,\limsup_{n\to\infty}b_n=B.$
    \begin{enumerate}
        \item 若$A=-\infty$或$B=+\infty$,则显然成立.
        \item 当$A=+\infty$时,由$\limsup_{n\to\infty}a_n=A$有$\forall \xi >A,\exists N_1\in\N^*\st\forall n>N:a_n>\xi.$由条件可知,$\exists N'=\max\{N,N_1\}\st\forall n>N':b_n\geqslant a_n\geqslant\xi$.
        这说明$B=+\infty$,从而成立$B\geqslant A$.类似可证,$B=-\infty$的情况.
        \item 当$A,B\in\R$时,用反证法.
        假设$A>B$,于是$\exists \varepsilon >0\st A>B+\varepsilon$.
        由$\limsup_{n\to\infty}b_n=B$,有$\forall \varepsilon >0,\exists N_2\in\N^*\st\forall n>N_2:b_n<B+\varepsilon$.
        于是$\exists N''=\max\{N,N_2\}\st\forall n>N'': a_n\leqslant b_n<B+\varepsilon<A$.这说明$(B+\varepsilon,+\infty)$内只有$\{a_n\}$有限项,则$A$不是极限点,这与$A$是上极限矛盾.因此假设不成立,于是$A\leqslant B.$
    \end{enumerate}
    综上所述,命题得证.
\begin{note}
    注意定义的$\varepsilon-N$语言和数列极限的邻域刻画(拓扑),即$a$任意邻域外只有$\{a_n\}$的有限项.
\end{note}
\end{proof}

任一数列总是有上下极限的,因此相比于数列极限,用上下极限可以简化处理问题.

\begin{example}[Stolz-Cesalo定理]%//TODO
    设数列$\{a_n\},\{b_n\}$,若$\{b_n\}$严格递增,且$\lim_{n\to\infty}b_n=+\infty$,则有
    \[\liminf_{n\to\infty}\frac{a_n-a_{n-1}}{b_n-b_{n-1}}\leqslant\liminf_{n\to\infty}\frac{a_n}{b_n}\leqslant\limsup_{n\to\infty}\frac{a_n}{b_n}\leqslant\limsup_{n\to\infty}\frac{a_n-a_{n-1}}{b_n-b_{n-1}}.\]
\end{example}

已经给出了两种定义上下极限的方式,它们比较直观,然而与数列没有“直接挂钩”,在处理某些问题是并不是很便利.下面尝试从新角度对上下极限进行定义.由于研究数列极限不需要关心前面有限项,我们定义以下数列.

\begin{definition}
    设数列$\{a_n\}$,令
    \begin{align*}
        &L_n\coloneq\sup_{k\geqslant n}a_k=\sup\{a_n,a_{n+1},\cdots\},\quad n=1,2,\cdots,\\
        &l_n\coloneq\inf_{k\geqslant n}a_k=\inf\{a_n,a_{n+1},\cdots\},\quad n=1,2,\cdots.
    \end{align*}
    则$\{L_n\}$单调递减,$\{l_n\}$单调递增.
\end{definition}

\begin{proof}
    只证$\{L_n\}$单调递减,$\{l_n\}$单调递增类似.
    \[\{a_k:k\geqslant n+1\}\subset\{a_k:k\geqslant n,\quad n=1,2,\cdots.\]
    此即得证.
\end{proof}

\begin{note}
    $\{L_n\}$为$\{a_n,a_{n+1},\cdots\},n=1,2,\cdots$的上确界数列,对应的$\{l_n\}$为下确界数列.
\end{note}

\begin{theorem}[数列的上极限和下极限:定义方式3]
    \label{sxjxdy3}
    设数列$\{a_n\}$,则
    \begin{align*}
        &(1)\quad\limsup_{n\to\infty}a_n=\lim_{n\to\infty}\z(\sup_{k\geqslant n}a_k\y),\\
        &(2)\quad\liminf_{n\to\infty}a_n=\lim_{n\to\infty}\z(\inf_{k\geqslant n}a_k\y).
    \end{align*}
\end{theorem}

\begin{proof}%//TODO
    只证明(1),类似可证(2).
    令$\limsup_{n\to\infty}a_n=L,\sup_{k\geqslant n}a_k=L_n.$
    则转化为证明$\lim_{n\to\infty}L_n=L.$
    \begin{enumerate}
        \item 当$L=+\infty$时,$\{a_n\}$有一个极限点为$+\infty$的子列,因此\[L_n=\sup\{a_n,a_{n+1},\cdots\}=+\infty,\quad n=1,2,\cdots.\]
        即无论$n$多大$L_n$的上确界总是$+\infty$.于是可知,$\lim_{n\to\infty}a_n=+\infty.$
        \item 当$L=-\infty$时,$\lim_{n\to\infty}a_n=-\infty.$
        故\[\forall G>0,\exists N\in\N^*\st\forall n>N:a_n<-G.\]
        于是\[L_N=\sup\{a_{N},a_{N+1},\cdots\}<-G,\quad n=1,2,\cdots.\]
        由于$L_n$单调递减,于是$n>N$时$L_n\leqslant L_{N}<-G.$这就说明,$\lim_{n\to\infty}=-\infty.$
        \item 当$L\in \R$时,
        \begin{enumerate}
            \item 证明$L\leqslant\lim_{n\to\infty}L_n.$\\
            任取$\{a_n\}$的一个极限点$l$,则存在一个子列$a_{k_i}$收敛于$l$.对给定的$n$选取$i\geqslant n,$则$k_i\geqslant i\geqslant n$.
            于是\[a_{k_i}\leqslant\sup\{a_n,a_{n+1},\cdots,a_{k_i},\cdots\}=L_n,\quad n=1,2,\cdots.\]
            先令$i\to +\infty$,再令$n\to +\infty,$得$l\leqslant\lim_{n\to\infty}L_n$.由于$l$是任取的,所以有$L\leqslant\lim_{n\to\infty}L_n.$
            \item 证明$L\geqslant\lim_{n\to\infty}L_n.$\\
            由\ref{sxjxdy2}可知,
            \[\forall \varepsilon >0,\exists N\in\N^*\st\forall n>N:a_n\leqslant L+\varepsilon.\]
            由于$L_n$单调递减,所以当$n>N$时,$a_n\leqslant a_N$,则$L_n\leqslant L_N\leqslant a_N\leqslant L+\varepsilon.$令$n\to\infty$,则有$\lim_{n\to\infty}L_n\leqslant L+\varepsilon$,又由于$\varepsilon$任意性(任意小),于是得$\lim_{n\to\infty}L_n\leqslant L.$
        \end{enumerate}
        于是$L=\limsup_{n\to\infty}L_n.$
    \end{enumerate}
    综上所述,证明了$\limsup_{n\to\infty}a_n=\lim_{n\to\infty}\z(\sup_{k\geqslant n}a_k\y)$
    \begin{note}\\
        1,2,3.(b)的证明思路是类似的,按定义方式2进行证明\\
        尤其注意3.(a)的证明中对子列项的控制
    \end{note}
\end{proof}

\begin{remark}
    数列的上下极限也可按此定义方法记作$\overline{\lim}a_n,\underline{\lim}a_n$.用这个定义方法处理关于上下极限的不等式较为简便.
\end{remark}

上下极限与极限的不同之处在于极限满足“可加性”,而上下极限分别满足“次可加性”和“超可加性”.

\begin{proposition}[上极限的次可加性和下极限的超可加性]
    设数列$\{a_n\},\{b_n\}$,则
    \begin{align*}
        &(1)\quad \liminf_{n\to\infty}a_n +\liminf_{n\to\infty}b_n\leqslant \liminf_{n\to\infty}(a_n+b_n)\leqslant \liminf_{n\to\infty}a_n+\limsup_{n\to\infty}b_n,\\
        &(2)\quad \liminf_{n\to\infty}a_n+\limsup_{n\to\infty}b_n\leqslant\limsup_{n\to\infty}(a_n+b_n)\leqslant\limsup_{n\to\infty}a_n+\limsup_{n\to\infty}b_n.
    \end{align*}
\end{proposition}

\begin{proof}
    只证明(1),类似可证(2).
    对任意$l\geqslant n$,有$\inf_{k\geqslant n}a_n\leqslant a_l,\inf_{k\geqslant n}b_n\leqslant b_l,$
    于是\[\inf_{k\geqslant n}a_k+\inf_{k\geqslant n}b_k\leqslant a_l+b_l.\]由于$l\geqslant n$的任意性,有
    \[\inf_{k\geqslant n}a_k+\inf_{k\geqslant n}b_k\leqslant \inf_{k\geqslant n}(a_k+b_k).\]
    于是可知,
    \[\inf_{k\geqslant n}a_k+\sup_{k\geqslant n}b_k\geqslant\inf_{k\geqslant n}(a_k+b_k)-\inf_{k\geqslant n}b_k+\sup_{k\geqslant n}b_k=\inf_{k\geqslant n}(a_k+b_k).\]
    于是
    \[\inf_{k\geqslant n}a_k +\inf_{k\geqslant n}b_k\leqslant\inf_{k\geqslant n}(a_k+b_k)\leqslant\inf_{k\geqslant n}a_k+\sup_{k\geqslant n}b_k.\]
    再令$n\to\infty$,由\ref{sxjxdy3}得,
    \[\liminf_{n\to\infty}a_n+\liminf_{n\to\infty}b_n\leqslant\liminf_{n\to\infty}(a_n+b_n)\leqslant\liminf_{n\to\infty}a_n+\limsup_{n\to\infty}b_n.\]
    此即得证.
\end{proof}

\begin{remark}
    上面的定理不等式两端需要有意义,不能出现$\infty -\infty$型.
\end{remark}

\begin{remark}
    设定义在$A$上的映射$f$.
    \begin{enumerate}
        \item 若$f(a+b)=f(a)+f(b),\forall a,b\in A$,则称$f$满足可加性(additivity).
        \item 若$f(a+b)\geqslant f(a)+f(b),\forall a,b\in A$,则称$f$满足超可加性(superadditivity).
        \item 若$f(a+b)\leqslant f(a)+f(b),\forall a,b\in A$,则称$f$满足次可加性(subadditivity).
    \end{enumerate}
\end{remark}%//INFO 数列的上极限和下极限%//INFO 数列的收敛判别法
