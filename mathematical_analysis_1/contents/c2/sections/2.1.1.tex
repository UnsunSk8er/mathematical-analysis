\subsection{无理数的历史}

\begin{example}
    证明二次方程$x^2=2$没有有理根.
    \begin{proof}
        即证明$\sqrt{2}$不是有理数.
        用反证法.假设$\sqrt{2}\in\Q$,则$\sqrt{2}=\frac{q}{p}$,其中$p,q\in\Z$互素且$p\ne 0$.从而
        \[\z(\frac{q}{p}\y)^2=2\iff q^2=2p^2.\]
        于是$q$为偶数,从而$q=2k,k\in\Z$,于是
        \[(2k)^2=2p^2\iff p^2=2k^2.\]
        这说明按假设$p$也是偶数,这与$p,q$互素矛盾.假设不成立,从而命题得证.
    \end{proof}
\end{example}

\begin{remark}
    这个例子说明,$\Q$中仍然有间隙,是不连续的.
\end{remark}

\begin{note}关于超越数与实数的构造
    
    虽然$\sqrt{2}$不是有理数,但它还是一个代数方程的根.而更多无理数是不能表示成代数方程的根的,比如圆周率$\pi$,因为代数方程终归是“有限的游戏”.虽然如此,代数方程可以表示许多无理数,“有限的游戏”构造出“无限的结果”,这是多么美妙.
    
    而像圆周率$\pi$这样能表示成代数方程的根的无理数,称为超越数,结合上面的讨论知道,它不能通过有限次运算得到.因此更多的无理数,像超越数不能用有限的过程来构造,只能通过无限的过程!这就表明,实数集的构造用整数集和有理数集的构造方法是行不通的.即若用整数集和有理数集的构造方法构造实数,是构造不出超越数的.
\end{note}

\begin{remark}
    上面的讨论说明,有限次的乘方逆运算,并不能囊括所有的无理数!
\end{remark}