\subsection{邻域和极限的简单性质}
\begin{definition}[邻域]
    给定$a\in\R$,令
    \[N_r(a)\coloneq\{x\in\R:|x-a|<r\},\quad \forall r\in \R^*.\]
    称$N_r(a)$为以$a$为中心,$r$为半径的邻域(neighbourhood),在不强调半径的情况下可记作$N(a)$.
\end{definition}

\begin{remark}
    在$\R$中,容易知道$N_r(a)=(a-r,a+r)$,然而我们没有用开区间直接定义,而是用度量对邻域进行定义.
\end{remark}

邻域在数学分析中用于描述“局部概念”十分便利.下面用邻域的观点叙述数列极限的定义.

\begin{theorem}
    设数列$\{a_n\}$,\\
    (1)$\{a_n\}$收敛于$a\in\R\iff n$充分大时$\{a_n\}$的各项都落在任一给定的邻域$N_{\varepsilon}(a)$中.\\
    (2)$\{a_n\}$收敛于$a\in\R\iff a$的任一邻域外都只有$\{a_n\}$的有限项.
\end{theorem}

\begin{proof}
    必要性按数列极限的定义显然成立.
    任取$\varepsilon>0$,则$N_{\varepsilon}(a)$外只有$\{a_n\}$的有限项,设其中下标最大的那个为$a_N$,则当$n>N$时,$\{a_n\}$全都落在了$N_{\varepsilon}(a)$中.此即按数列极限定义得证.
\end{proof}

\begin{remark}
    (1)注意,若$a$的任一邻域$N(a)$内都有$\{a_n\}$的无穷多项,不能推出$\{a_n\}$收敛.只能说有极限点.
    (2)可视为数列极限的“拓扑定义”.
\end{remark}

邻域是一个拓扑概念,从邻域的角度思考可以使问题更加直观.极限是一个无穷的概念,上述定理的第(2)条,使得我们将无穷问题转化为有穷问题来解决,这使得问题变得简单易操作了.