\subsubsection{序偶}

这一小子节介绍元素有顺序的双元素集.

我们已经从配对公理获得了双元素集,但是由外延公理我们知道有下面的事实:
\[\{x,y\}=\{y,x\}.\]
这说明集合里的元素是没有先后顺序的.

而为了使集合含有顺序的信息,我们尝试一些定义一些有序的二元集.

\begin{example}
\begin{align*}
    &\langle x,y\rangle_1=\{x,y\},\\
    &\langle x,y\rangle_2=\{x,\{y\}\}.
\end{align*}
容易验证这两个尝试定义是失败的,即存在集合$x,y$使得上面的两个"序对"成立$\langle x,y\rangle=\langle y,x\rangle$.    
\end{example}

下面我们正式定义序对.
\begin{definition}
    [序偶]
    \[
    \langle x,y\rangle\coloneq\{\{x\},\{x,y\}\}\]
    称上述定义的$\langle x,y\rangle$为序偶或序对.
\end{definition}
\begin{note}
    这个定义由Kazimierz Kuratowski在1921年引入,在今天被广泛使用.另外,第一个定义成功的序偶由Norbert Wiener在1914年引入,定义如下:
    \[\langle x,y\rangle_3\coloneq\{\{\{x\},\emptyset\},\{\{y\}\}\}.\]
    它可以由下面的定理判定.
\end{note}

\begin{theorem}
    [序偶的判定]
    \[\langle u,v\rangle=\langle x,y\rangle\iff (u=x)\wedge (v=y).\]
\end{theorem}

\begin{proof}%//TODO
\end{proof}
正是由于有了这个我们想要的性质,序偶才被称为序偶.它告诉我们$\langle x,y\rangle$的第一个坐标(coordinate)为$x$,第二个坐标为$y$.

于是我们可以通过序偶引入笛卡尔积(Cartesian product).
\begin{definition}
    [笛卡尔积]
    \[A\times B\coloneq\{\langle x,y\rangle\mid x\in A \wedge y\in B\}\]
    上述定义的类$A\times B$称为$A$和$B$的笛卡尔积.
\end{definition}

为了验证笛卡尔积为一个集合,我们有以下引理和推论.

\begin{lemma}
    \[(x\in C)\wedge(y\in C)\implies\langle x,y\rangle\in\p\p C.\]
\end{lemma}
\begin{proof}
    %//TODO
\end{proof}

\begin{corollary}
    \[\forall A,B\exists C[x\in C\iff x\in A\times B]\]
\end{corollary}
\begin{proof}
    从子集公理模式我们可以构造$C$如下,
    \[\{w\in\p\p(A\cup B)\mid \exists x\in A,y\in B(w=\langle x,y\rangle)\}.\]由引理可知$C$包含了所有这样形式的序偶.
\end{proof}
序偶还有其他定义,只要满足序偶的判定条件即可.

\begin{note}
    这一小子节的序偶的存在性由配对公理保证,唯一性由外延公理保证;笛卡尔积的存在性由子集公理保证,唯一性由外延公理保证.
\end{note}

    