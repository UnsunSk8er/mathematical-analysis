\subsubsection{公理化方法}

这一小子节介绍集合论的公理化方法.

公理化的方法和好处是:原始假设完全明确,所有的定理都来自于这些公理,且可被证明.任何数学对象可被定义为某些特定集合,而定理则是被视为对这些集合的描述,这些定理都可由公理推出。因此公理给出了一系列足以推动整个数学发展的前提假设。

\begin{note}
    数学可以嵌入到公理化集合论中,即数学研究完全依靠于公理化集合论.
\end{note}

没有被公理化的集合论称为朴素集合论(naive set theory)。朴素集合论本身的悖论推动了公理集合论化的发展。尽管没有朴素集合论导致的悖论,公理化集合论还是要处理特定原理的真假,比如分离公理。并且公理化过程中我们选择的公理要尽量准确地符合我们对集合和类的朴素看法。

我们的公理系统开始于两个原始观念,即集合和元素,它们仍然是未定义的。取而代之,我们用一些公理来描述它们.

\begin{note}
    可以认为这些公理描述了集合和元素的某些性质,以此“定义”了集合和元素.
\end{note}

这一系列公理确立采用之后,我们将去探讨它们的逻辑结论,即定理.陈述$\sigma$是定理若任何使公理成立的对未定义的集合和元素观念的意义赋予使得陈述$\sigma$成立。

\begin{note}
    定理的成立不依赖于集合和元素的概念,只依赖于公理.
\end{note}

如果$\sigma$是一系列公理的定理,那么$\sigma$总是有一个从公理出发有限长的证明.

\begin{example}
    将外延公理作为我们的第一个公理,那么$\emptyset$是唯一的是一个定理.
\end{example}
\begin{note}
    这个定理的证明不依赖于集合和元素的概念,依赖于逻辑术语如每个、没有、等于等等.
\end{note}

\begin{example}
    $\sigma$:有两个集合,其中一个是另一个的元素.$\sigma$不是外延公理的定理.

    令"集合"表示"等于$2$的数",令"是$\cdots$的元素"表示"不等于".则外延公理真,而$\sigma$假.
\end{example}

\begin{note}
    由上面的例题知道,对于$\sigma$我们需要“好”的意义赋予.
\end{note}





