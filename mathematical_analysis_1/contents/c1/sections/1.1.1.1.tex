\subsubsection{Baby集合论}
这一小子节介绍朴素集合论中外延原理及集合的相等、空集、元素集、集合的并与交、子集、幂集、集合的表示及概括公理模式.
\begin{note} 集合论讨论的对象是集合.我们始终关注的是集合.后面我们将看到Baby集合论中讨论的对象其实是类.在公理化集合论中我们将区分集合和类的概念.

    集合(set),可以简单理解为一个不计大小的篮子,它可以装东西,也可以不装。如果装东西,那么里面装的东西可以粗略分为篮子即集合,和其他东西比如苹果、数字等等.不久我们将知道装的东西相同的篮子是同一个篮子.
\end{note}

集合是一系列东西的采集,这些东西称为这个集合的成员(member)或元素(element).这一系列东西被看作一个单独的对象,即集合。若$t$是集合$A$的一个元素,写作

\[t\in A.\]

若$t$不是集合$A$的一个元素,写作

\[t\notin A.\]

$\in$读作属于.
需要指出的是$t$可以是包含集合在内的一切东西.
\begin{example}
    \[1\in\{1\},\quad
    \{1\}\notin\{1\}.\]
\end{example}
若集合$A$和集合$B$有相同的元素,即$A$中的任意元素也是$B$中的元素,则称集合$A$和集合$B$是相等的.这被称为外延原理.

\begin{proposition}
    [外延原理]
    若集合$A$和集合$B$,对任意的对象$t$,$t$在集合$A$中当且仅当$t$在集合$B$中,即
    \[t\in A \iff t\in B,\]
    则$A=B$,否则$A\ne B$.
\end{proposition}

反过来(converse),若$A=B$,则$t\in A\iff t\in B.$

\begin{example}
    \[1\ne\{1\},\quad \{1\}=\{1\}.\]
\end{example}

\begin{note}
    外延原理保证了一些集合若存在则唯一,刻画的是集合的唯一性。
    而集合的存在完全决定于集合的元素或者元素的性质,因而集合存在性的刻画是由元素的罗列或者是对元素性质的描述.这对应了集合的两个表示法,即列举法和描述法,其实描述法包含了列举法,只要描述一个集合的元素是确切的哪些,即确定了集合,然而这种描述法的直观印象会出问题,后面将会讨论.
\end{note}

上面的讨论似乎都是对有元素的集合,是否存在不含任何元素的集合?用描述法即可构造这样的集合,只要把这个集合描述为:不含任何元素的集合.这样的集合称为空集,记作$\emptyset$.

\begin{note}
    空集的存在性由性质描述保证,空集的唯一性由外延公理保证。
\end{note}

若将空集作为元素可以构造一个新集合$\{\emptyset\}$,像这样只有一个元素的集合称为单元素集.若再将这个集合作为元素可以构造另一个新集合$\{\{\emptyset\}\}$,这样下去可以构造一列集合:

\[
\emptyset,\{\emptyset\},\{\{\emptyset\}\},\cdots .   
\]

取出前面几个,比如取出前$3$个,那么它们可以构成一个新的集合$\{\emptyset,\{\emptyset\},\{\{\emptyset\}\}\}$,这是个三元素集合.

\begin{note}
    这告诉我们不管有多少集合,每个集合有没有元素,有多少元素,总能从它们构造出新的更“大”的集合。
\end{note}

关于这个三元素集合有另一种重要的看法,即它的元素是上述构造的一列集合的第二个、第三个、第四个这三个集合中的所有元素.这就引出了并集的概念,$A$和$B$的并集记作$A\cup B$,它包含$A,B$的所有元素,以上面的三元素为例:

\begin{example}
    \[
    \{\emptyset,\{\emptyset\},\{\{\emptyset\}\}\}=\{\emptyset\}\cup\{\{\emptyset\}\}\cup\{\{\{\emptyset\}\}\}.
\]
\end{example}

集合的并(union)可以看作是是集合间的一种运算(operation).$A$和$B$的并$A\cup B$中的元素的描述也可以叙述为属于$A$或者属于$B$或者同时属于$A,B$的元素.集合间另一种熟悉的运算是集合间的交(intersection).$A$和$B$的交记作$A\cap B$,其中的元素对应并运算可叙述为属于$A$并且属于$B$的元素.

\begin{example}
    \[
    \emptyset\cup\{\emptyset\}=\{\emptyset\},\quad \emptyset\cap\{\emptyset\}=\emptyset.    
    \]
\end{example}
\begin{note}
    可以看出并产出的集合比交产出的集合更"大".
\end{note}

\begin{note}
    上面的讨论都是从集合的元素本身入手去构造新的集合,即使是集合间的运算.下面引入子集的概念,以便于讨论集合间的关系和构造更大的集合.
\end{note}

若$A$的所有元素都是$B$的元素,则称$A$是$B$的子集,记作$A\subseteq B$,读作$A$包含于$B$.集合间的这种关系称为包含(inclusion)关系,注意这要区别于元素和集合间的属于(membership)关系.

\begin{example}
    \[
        \emptyset\in\{\emptyset\},\quad
        \emptyset\subseteq\{\emptyset\},\quad \{\emptyset\}\notin\{\emptyset\},\quad \{\emptyset\}\subseteq\{\emptyset\}.
    \]
\end{example}

\begin{example}
    \[A\subseteq B \quad\textit{且}\quad B\subseteq A\iff A=B\]
\end{example}

空集是任何集合的子集,因此任何集合至少有一个子集.更确切地说,除空集外的所有集合至少有两个子集.事实上,一个含有$n$个元素的集合有$2^n$个子集.

将一个集合$A$的所有子集拿过来,立即可以构成一个新的集合,称为$A$的幂集(power set),记作$\mathscr{P}A$.

\begin{example}
    \[\mathscr{P}\emptyset=\{\emptyset\},\quad
    \p\{\emptyset\}=\{\emptyset,\{\emptyset\}\},\quad
    \p\{\emptyset,\{\emptyset\}\}=\{\emptyset,\{\emptyset\},\{\{\emptyset\}\},\{\emptyset,\{\emptyset\}\}\}.\]
    讨论这三个集合的属于关系和包含关系.
\end{example}

关于集合的表示,前面提到有列举法和描述法(abstraction)。描述法形如:
\[\{x\mid\_\_x\_\_\}.\]
朴素观点的并集和交集都可以写成这种形式:
\begin{gather*}
    A\cup B=\{x:x\in A\textit{或}x\in B\},\\
    A\cap B=\{x:x\in A\textit{且}x\in B\}.
\end{gather*}
描述法来自于我们对集合的朴素的直观认识,即我们倾向于认为集合有描述法的形式,也就是它可以被表述为以下形式:
\begin{axiom}
    [概括公理模式Comprehension Axioms]
    若$P$表示一种性质,则存在集合\[Y=\{x:P(x)\}.\]
\end{axiom}
上述讨论的朴素集合论中得到的集合普遍都是由概括公理模式得到的。
但它存在两个潜在风险,第一个风险来自于性质描述本身,这里不讨论。
第二个风险则是由于:在Baby集合论本身的公理化过程中,我们应当始终认为集合论的对象是集合,而概括公理模式不仅包括了集合还包括了类,即我们没有区分集合和类,这会导致一些悖论。
\begin{example}[罗素悖论(Russel's Parodox)]令
    \[S=\{x\mid x\notin x\}.\]
    若$S\in S$,则通过$S$的的性质描述$S\notin S$,这就产生了矛盾。
\end{example}
为了避免这类悖论,因此才引入了类(class)的概念以此便于讨论。也就是说,我们朴素观点下描述法产生的集合,形如
\[\{x:P(x)\}\]
应该都称作是类,所有的集合都是类,而类并不一定是集合,不是集合的类我们称为真类(proper class)。
\begin{example}
    令
    \[S=\{x:x\in x\},\]
    则$S$是一个真类,即不存在包含所有集合的集合。这将在后面给出证明.
\end{example}
随着集合和类的区分,在公理化集合论中这种风险被避免,从而公理揭示的对象排除了真类,只剩下集合。尽管描述法会带有风险,但在公理化集合论下用描述法表示集合仍然是方便的,只是使用时需要明白描述法表示的类是不是集合.
\begin{note}
    由于集合论的讨论的对象为集合,因此在我们的公理化集合论中,这个公理会将被子集公理所代替以避开真类.
\end{note}




