\subsubsection{集合与类——朴素观点}
前面讲述了朴素集合论观点下的集合其实都是类,因此讨论的所有东西都是关于类的。

这一小子节将说明集合的特点,以此较为直观地区别集合和类。

下面我们将用非正式的观点讨论如何得到一些集合.首先我们引入原子(atom)的概念.
原子是那些自身不是集合但我们想让它们作为集合的元素的东西.

\begin{note}
原子可以理解为可以装进篮子,但自身不涉及篮子的所有东西.
\end{note}
令$A$包含所有的原子,$A$在我们的描述下的第一个集合.
现在可以建立一个阶梯:
\[V_0\subseteq V_1\subseteq V_2\subseteq \cdots\]
取$V_0=A$,即$V_0=\{atoms\}$.那么下一层会包含所有原子组成的集合$V_0$:
\[V_1=V_0\cup\p V_0=A\cup\p A.\]
第三层包含所有低层的元素和所有低层元素构成的集合:
\[V_2=V_1\cup\p V_1.\]
并且一般地有:
\[V_{n+1}=V_n\cup\p V_n.\]
于是我们得到了接续的$V_0,V_1,V_2,\cdots$.但即使是这样一个无穷的阶梯,也没有包含足够多的集合.

\begin{example}
    \[\{\emptyset,\{\emptyset\},\{\{\emptyset\}\},\cdots\}\]不包含在上面的阶梯中.
\end{example}

为了弥补这个缺陷,取
\[V_{\omega}=V_0\cup V_1\cup\cdots,\]
然后令
\[V_{\omega+1}=V_{\omega}\cup\p V_{\omega},\]
一般地,对任意的$\alpha$,
\[V_{\alpha+1}=V_{\alpha}\cup\p V_{\alpha},\]
这个过程可以持续做下去。

对每个集合$a$,存在某个$\alpha$,使得$a\in V_{\alpha+1}$。$\alpha$称为这个集合的阶(rank).这就是集合的本质,集合一定在上述阶梯中的某处.如果要判定一些东西的采集是否是集合,则就是要判断它是否在上述的阶梯中.若它不在这个阶梯中,那么它就不是集合,真类则用来表示这些不是集合的“集合“.

\begin{note}
 集合的阶数为括号对数减去$1$.
\end{note}

由于在集合论中讨论的对象是集合,我们排除考虑原子,原子在数学中没有明确的必要目的。取$A=\emptyset$,这样我们排除了原子,简化了阶梯.
\[V_{\alpha+1}=V_{\alpha}\cup\p V_{\alpha},\]
被简化为了
\[V_{\alpha+1}=\p V_{\alpha}.\]

\begin{note}
    这样简化与后续内容序数有关。
\end{note}
