\subsubsection{集合代数}
这一小子节讨论集合上的包含关系和交并差运算.

并运算由并集公理得到:

\[A\cup B=\{x\mid x\in A\vee x\in B\}\]

包含关系(inclusion relation)、交(intersection)、差(relative complementation)由子集公理模式得到:

\begin{align*}
    &A\subseteq B\iff x\in A\implies x\in B,\\
    &A\cap B=\{x\mid x\in A\wedge x\in B\},\\
    &A-B=\{x\in A\mid x\in B\},\\
\end{align*}
补的概念由包含关系和差运算共同得到,若$A\subseteq B$,则$B-A$为$A$在$B$中的补,可以记作$C_B^A$,若讨论的集合都在$B$中则可以记作$-A$或$A^c$:
\[A^c=\{x\in A\mid x\notin B\}.\]
此时$B$称为空间(space).集合的补又称相对补,本质上是差运算.

一个集合$A$的绝对补是类不是集合:
\[A^c=\{x\mid x\notin B\}.\]
即绝对补不在一个指定的集合中讨论,不能由子集公理模式得出.

集合的包含关系、交并差运算称为集合的代数.下面是集合代数的一些基本事实.
\begin{proposition}
    [集合代数恒等式]
    结合律(Associative laws):\[A\cup(B\cup C)=(A\cup B)\cup C,\quad A\cap(B\cap C)=(A\cap B)\cap C.\]
    交换律(Commutative laws):\[A\cup B=B\cup A,\quad A\cap B=B\cap A.\]
    分配律(Distributive laws):\[A\cap(B\cup C)=(A\cap B)\cup(A\cap C),\quad A\cup(B\cap C)=(A\cup B)\cap(A\cup C).\]
    德摩根律(De Morgan's laws):\[C-(A\cup B)=(C-A)\cap(C-B),\quad C-(A\cap B)=(C-A)\cup(C-B).\]
    涉及空集的恒等式(Identities involving $\emptyset$):\[A\cup\emptyset=A,\quad A\cap\emptyset=\emptyset,\quad A\cap(C-A)=\emptyset.\]
\end{proposition}
上面的所有规律对一般集合都成立.而对于一些特殊的集合,其中一些规律的形式会简化,下面进行讨论.

若空间确定为$S$,则德摩根律会变为:
\[-(A\cup B)=-A\cap -B,\quad -(A\cap B)=-A\cup -B.\]
进一步有:
\[A\cup S=S,\quad A\cap S=A,\quad A\cup S=S,\quad A\cap-A=\emptyset.\]

现在来谈论如何证明这些法则,一共有两种方法:韦恩图和真值表.

对于包含关系,我们有以下单调性和反单调性:
\begin{align*}
    &A\subseteq B\implies A\cup C\subseteq B\cup C,\\
    &A\subseteq B\implies
    A\cap C\subseteq B\cap C,\\
    &A\cup B\implies C-A\supseteq C-B,\\
    &A\cup B\implies \bigcup A\subseteq \bigcup B,\\
    \emptyset\ne &A\cup B\implies\bigcap A\supseteq\bigcap B.
\end{align*}
前面三条涉及三个集合,后面两条只涉及两个集合和任意并,下面讨论涉及任意并和任意交的运算律.

\begin{proposition}
    [分配律(Distributive laws for $\mathscr{B}\ne\emptyset$)]
\begin{align*}
    &A\cup\bigcap\mathscr{B}=\bigcap\{A\cup X\mid X\in\mathscr{B}\},\\
    &A\cap\bigcup\mathscr{B}=\bigcup\{A\cap X\mid X\in\mathscr{B}\}.
\end{align*}
\end{proposition}

\begin{note}
    命题中的集合有其他记号
    \[\bigcap_{X\in\mathscr{B}}(A\cup X)=\bigcap\{A\cup X\mid X\in\mathscr{B}\},\quad \mathscr{B}\ne\emptyset,\]
\end{note}

记号$\{A\cup X\mid X\in\mathscr{B}\}$是描述法的拓展,它确定了集合$\mathscr{D}$:
\[t\in\mathscr{D}\iff \exists X\in\mathscr{B}\st t=A\cup X.\]
$\mathscr{D}$的存在性由子集公理模式确保,首先观察到$A\cup X\subseteq A\cup\bigcup\mathscr{B}$,从而$\mathscr{D}\subseteq\mathscr{P}(A\cup\bigcup\mathscr{B})$,
于是
\[\mathscr{D}=\{t\in\mathscr{P}(A\cup\bigcup\mathscr{B})\mid \exists X\in\mathscr{B}\st t=A\cup X\}.\]
相比之下,记号
\[\mathscr{D}=\{A\cup X\mid X\in\mathscr{B}\}\]更为简洁.

\begin{example}%//TODO
    \[\{C-X\mid X\in\mathscr{A}\}\]
\end{example}

\begin{proposition}
    [德摩根律(De Morgan's laws for $\mathscr{A}\ne\emptyset$)]
    \begin{align*}
        &C-\bigcup\mathscr{A}=\bigcap\{C-X\mid X\in\mathscr{A}\},\\
        &C-\bigcap\mathscr{A}=\bigcup\{C-X\mid X\in\mathscr{A}\}.
    \end{align*}
\end{proposition}

\begin{proof}
    \begin{align*}
        t\in C-\bigcup\mathscr{A}
        &\iff t\in C \wedge\neg(\exists x\in\mathscr{A}\st t\in x)\\
        &\iff t\in C-X,\forall X\in\mathscr{A}\\
        &\iff t\in\bigcap\{C-X\mid X\in\mathscr{A}\}.
    \end{align*}
\end{proof}

\begin{note}
    命题中的集合有其他记号
    \[\bigcup_{X\in\mathscr{A}}(C-X)=\bigcup\{C-X\mid X\in\mathscr{A}\}\]
\end{note}