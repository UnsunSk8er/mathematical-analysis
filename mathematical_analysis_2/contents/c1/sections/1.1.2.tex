\subsection{比较判别法}

为排除符号影响,先讨论"正项级数".

\begin{definition}[正项级数]
    设级数$\sum_{n=1}^\infty a_n$,当$n$充分大时,$a_n\geqslant 0$,则该级数称为正项级数(series of prositive terms).
\end{definition}

\begin{remark}
    由于去掉或加上级数前面有限项不影响级数敛散性,所以对前面有限项符号没有要求.
\end{remark}

\begin{note}
    类比开始研究积分时,积分函数为正.
\end{note}

$n$充分大时,正项级数部分和数列$S_n$单调递增,故正项级数要么收敛要么趋于正无穷.由单调有界收敛定理,反过来若级数部分和数列单调递增,则该级数一定是正项数列.

\begin{proposition}[正项级数收敛的充要条件]
    正项级数$\sum_{n=1}^\infty a_n$收敛当且仅当它的部分和数列$\{S_n\}$有界.
\end{proposition}

曾经在证明调和级数发散时,首先就是证明它无界.

\begin{example}%//TODO
    
\end{example}

证明数列有界时,基本思路是放大;反之证明数列无界时,是缩小.这个方法在判断级数敛散性时很有用.引出级数敛散比较判别法.

\begin{theorem}[比较判别法]
    设正项级数$\sum_{n=1}^\infty a_n,\sum_{n=1}^\infty b_n$,当$n$充分大时,有$a_n\leqslant b_n$.
    \begin{enumerate}
        \item 若$\sum_{n=1}^\infty b_n$收敛,则$\sum_{n=1}^\infty a_n$收敛.
        \item 若$\sum_{n=1}^\infty a_n$发散,则$\sum_{n=1}^\infty b_n$发散.
    \end{enumerate}
\end{theorem}

\begin{proof}%//TODO
    
\end{proof}

\begin{remark}
    以上定理只对正项级数成立.
\end{remark}

使用比较判别法时,关键是找到一个可比较的$p$级数.

\begin{note}
    比较判别法的关键在于不等式的放缩,而这对技巧性的要求更高.因此想到用极限来转化不等式放缩问题.
\end{note}

\begin{note}
    数学分析中,粗略的讲,一种东西可以写成三种形式:
    \begin{enumerate}
        \item 不等式形式
        \item 极限形式
        \item Landou记号
    \end{enumerate}
    而从不等式到Landou记号越来越粗略.
\end{note}

\begin{theorem}[比较判别法的极限形式]
    设正项级数$\sum_{n=1}^\infty a_n,\sum_{n=1}^\infty b_n.$若
    \[\lim_{n\to\infty}\frac{a_n}{b_n}=l.\]
    则
    \begin{enumerate}
        \item $0<l<+\infty$,$\sum_{n=1}^\infty a_n,\sum_{n=1}^\infty b_n$同敛散.
        \item $l=0$,若$\sum_{n=1}^\infty b_n$收敛,则$\sum_{n=1}^\infty a_n$也收敛.
        \item $l=+\infty$,若$\sum_{n=1}^\infty a_n$发散,则$\sum_{n=1}^\infty b_n$也发散.
    \end{enumerate}
\end{theorem}

\begin{proof}%//TODO
    
\end{proof}

\begin{note}
    $\{a_n\},\{b_n\}$都是无穷小.
\end{note}

以上定理说明,判断正项级数敛散性,转化为找级数通项的同阶无穷小.

\begin{example}
    判断以下级数敛散性:
    \[\sum_{n=1}^\infty (1-\cos\frac{x}{n}).\]
\end{example}

\begin{remark}
    此例表明,级数可以定义一个函数.
\end{remark}

比较判别法不仅可以直接用于判断正项级数的敛散性性, 还可以派生出很多别的判别法. 如果以几何级数为 比较对象, 可以得到所谓的根值判别法 (root test).

\begin{theorem}[Cauchy根值判别法]
    
\end{theorem}

\begin{theorem}[Cauchy根值判别法的极限形式]
    
\end{theorem}

如果以 $p$ 级数为比较对象, 可以得到所谓的对数判别法 (logarithmic test).

\begin{proposition}[对数判别法]
    
\end{proposition}

\begin{proposition}[对数判别法的极限形式]
    
\end{proposition}

