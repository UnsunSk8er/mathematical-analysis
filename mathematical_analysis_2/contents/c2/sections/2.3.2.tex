\subsection{一般的范数和内积}

\begin{axiom}
    [范数]%//TODO
    设线性空间$V$.定义一个映射
    \[\lVert \cdot\rVert:V\to\R\]
    $\lVert \cdot\rVert$满足:
    \begin{enumerate}
        \item 正定性
        \item 绝对齐次性
        \item 三角不等式
    \end{enumerate}则称$\lVert \cdot\rVert$是$V$中的一个范数.定义了范数的线性空间$V$称为赋范线性空间(normed vector space),简称赋范线性空间(normed space),记作$(V,\lVert\cdot\rVert)$.
\end{axiom}

\begin{remark}
    范数也是一个泛函.
\end{remark}

\begin{remark}
    齐次性是范数和度量的本质差别.
\end{remark}

\begin{note}
    一般不可以在一般域上定义范数.
\end{note}


\begin{theorem}
    [范数诱导的度量]
    在赋范空间$(V,\lVert\cdot\rVert)$中令
    \[d(\alpha,\beta)=\lVert\alpha -\beta\rVert,\]
    则$d$是$V$的一个度量.
\end{theorem}

\begin{remark}
    定理表明任一范数可以定义一个度量,于是这个赋范空间一定可以成为一个度量空间.反之不然,有些度量是无法用范数诱导.
\end{remark}

\begin{remark}
    范数诱导的度量必满足:
    \begin{enumerate}
        \item 平移不变性:$d(\alpha)=d(\alpha+x,\beta+x)$
        \item 绝对齐次性:$d(k\alpha,k\beta)=|k|d(\alpha,\beta)$
    \end{enumerate}
\end{remark}

\begin{axiom}
    [内积]%//TODO
    设实线性空间$V$.定义一个映射:
    \[(\cdot,\cdot):V\times V\to\R,\]
    若满足
    \begin{enumerate}
        \item 正定性
        \item 对称性
        \item 双线性
    \end{enumerate}
    则称$(\cdot,\cdot)$为$V$上的一个内积(inner product).定义了一个内积$(\cdot,\cdot)$的集合$V$称为内积空间(inner product space),记作$(V,(\cdot,\cdot))$.
\end{axiom}

\begin{remark}
    一般的线性空间对称性可以放宽为“共轭对称性”,此是为了让复数域上的线性空间,即酉空间上的内积.
\end{remark}

\begin{remark}
    内积也是一个泛函.
\end{remark}

\begin{theorem}
    [Cauchy-Schwartz不等式]
    设线性空间$V,(\cdot,\cdot)$中有
    \[(\alpha,\beta)^2\leq (\alpha,\alpha)(\beta,\beta),\quad\forall\alpha,\beta\in V,\]
    等号成立当且仅当$\alpha,\beta$线性相关.
\end{theorem}

\begin{theorem}
    [内积诱导的范数]
    内积空间$(V,(\cdot,\cdot))$中,令
    \[\lVert\alpha\rVert=\sqrt{(\alpha,\alpha)}\]
    则称$\lVert\cdot\rVert$是$V$的一个范数.
\end{theorem}

\begin{remark}
    定理表明任一内积可以定义一个范数,于是这个内积空间必然可以成为一个赋范空间,从而成为一个度量空间.
\end{remark}

\begin{remark}
    内积诱导的范数必满足平行四边形等式:
    \[\lVert\alpha-\beta\rVert^2+\lVert\alpha+\beta\rVert^2=2(\lVert\alpha\rVert^2+\lVert\beta\rVert^2).\]
\end{remark}

\begin{example}
    [$L^p$范数和$L^p$度量]
    $\R^n$中,设$\alpha=(a_1,\cdots,a_n),\beta=(b_1,\cdots,b_n)$.令
    \[\lVert\alpha\rVert_p\coloneq\z(\sum_{i=1}^n|a_i|^p\y)^{1/p},\quad p\geqslant 1.\]
    则$\lVert\cdot\rVert_p$是$\R^n$上的一个范数.从而可以用$\lVert\cdot\rVert_p$定义一个度量:
    \[d_p(\alpha,\beta)=\lVert\alpha-\beta\rVert_p.\]
\end{example}

\begin{example}
    [$L^{\infty}$范数和$L^{\infty}$度量]
    $\R^n$中,设$\alpha=(a_1,\cdots,a_n),\beta=(b_1,\cdots,b_n)$.令
    \[\lVert\alpha\rVert_\infty\coloneq\max\{|a_1|,\cdots,|a_n|\}\]
    则$\lVert\cdot\rVert_p$是$\R^n$上的一个范数.从而可以用$\lVert\cdot\rVert_\infty$定义一个度量:
    \[d_\infty(\alpha,\beta)=\lVert\alpha-\beta\rVert_\infty.\]
\end{example}

\begin{proposition}
    [$\R^n$中范数和度量的常用不等式]
    $\R^n$中
    \begin{enumerate}
        \item $\lVert\cdot\rVert_\infty\leqslant\lVert\cdot\rVert_2\leqslant\lVert\cdot\rVert_1$
        \item $d(\cdot,\cdot)_\infty\leqslant d(\cdot,\cdot)_2\leqslant d(\cdot,\cdot)_1$
    \end{enumerate}
\end{proposition}