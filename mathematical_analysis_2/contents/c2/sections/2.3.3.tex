\subsection{度量空间的完备化}

在讨论$\R$的完备性的时候有$7$个等价命题:
\begin{enumerate}
    \item Dedekind定理
    \item LUB
    \item Hiene-Borel定理
    \item 单调有界收敛定理
    \item 闭区间套定理
    \item Bolzano-Werestrass定理
    \item Cauchy收敛原理
\end{enumerate}
现在想要在一般的度量空间中推广完备性.但由于一般的度量空间甚至$\R^n$空间中不具备序结构,从而Dedekind定理、LUB、单调有界收敛定理都难以推广,而其中直接涉及度量的定理只有Cauchy收敛原理.
于是可以考虑将它推广到一般的度量空间.

\begin{definition}
    [Cauchy列]
    度量空间(X,d)中,设点列$\{x_n\}$.若$\forall\varepsilon >0,\exists N\in\N^*\st\forall n>N:$
    \[d(x_m,x_n)<\varepsilon,\]
    则称该点列是一个Cauchy列(Cauchy sequence)或基本列(fundamental sequence).
\end{definition}

一般的度量空间中,Cauchy列未必收敛,比如$\Q$中的Cauchy列.

\begin{definition}
    [完备的度量空间]
    设度量空间$(X,d)$.若其中任一Cauchy列都收敛,则称$(X,d)$是完备的度量空间(complete metric space),简称完备空间(complete sapce).
\end{definition}

\begin{remark}
    完备的内积空间称为Hilbert空间,完备的赋范空间称为Banach空间.
\end{remark}

一般的度量空间没有Cauchy收敛原理,但Cauchy收敛原理的必要性仍然是成立的.

\begin{proposition}
    度量空间$(X,d)$中,收敛的点列必为Cauchy列.
\end{proposition}

尽管Cauchy列未必收敛,但一定有界,若有收敛子列,则必收敛.

\begin{proposition}
    在度量空间$(X,d)$中,Cauchy列一定有界.
\end{proposition}

\begin{proposition}
    度量空间$(X,d)$中,存在收敛子列的Cauchy列必收敛.
\end{proposition}

\begin{theorem}
    [$\R^n$中的Cauchy收敛原理]
    $\R^n$在$d_n$下是完备的,即在$d_n$下$\R^n$中的点列$\{x_k\}$收敛当且仅当它是Cauchy列.
\end{theorem}