\subsection{$\R^n$空间的线性结构}

\begin{definition}[有限维实空间$\R^n$]
    $n$元有序实数组集记作$\R^n$,这些数组称为向量(vector).其中$a_i$为向量$\alpha$的第$i$个分量(component).规定$\R^n$上的加法运算:
    \[(a_1,\ldots,a_n)+(b_1,\ldots,b_n)\coloneq (a_1+b_1,\ldots,a_n+b_n),\]
    再规定$\R^n$中元素与$\R$中实数间的数乘运算(scalar multiplication):
    \[k(a_1,\ldots,a_n)\coloneq (ka_1,\ldots,ka_n).\]
\end{definition}

\begin{remark}(Einstein notation)
    由于要研究$\R^n$中的点列,将$\R^n$中的点记作
    \[x=(x^1,\ldots,x^n).\]
    注意这种记法与幂指数的区别,当分量有次方时要加括号.
\end{remark}

\begin{proposition}[$\R^n$的线性性质]%//TODO

\end{proposition}

\begin{axiom}[线性空间]%//TODO
    设非空集合$V$和域$F$.定义$V$上的加法运算:
    \[+:V\times V\to V,\]
    $V$关于$F$的数乘运算:
    \[\cdot :F\times V\to V.\]
    加法运算满足$(V,+)$是个Abel群.
    数乘运算满足:
    \begin{enumerate}
        \item $\exists 1\in F\st 1 \cdot\alpha=\alpha,\quad\forall\alpha\in V$.
        \item $(kl)\alpha=k(l\alpha),\quad\forall k,l\in F,\alpha\in V$.
        \item $(k+l)\alpha=k\alpha+l\alpha,\quad\forall k,l\in F,\alpha\in V.$
        \item $k(\alpha+\beta)=k\alpha+k\beta,\quad\forall k\in F,\alpha,\beta\in F.$
    \end{enumerate}
    则称$V$是域$F$上的一个线性空间(linear space),或向量空间(vector space).
\end{axiom}

\begin{definition}[线性相关性]%//TODO
    
\end{definition}

\begin{definition}[$\R^n$的基]%//TODO
    
\end{definition}